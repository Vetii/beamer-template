\documentclass[mathserif]{beamer}
\usepackage{fontspec}
\usepackage{minted} % Source code listings
\usepackage{graphicx} % Adding figures
\usepackage{tabularx} % pretty tables

% Minted styles
\usemintedstyle{monokai}

% Beamer config
\setbeamerfont{title}{family=\fontspec{Roboto Slab Light}}
\setbeamerfont{frametitle}{family=\fontspec{Roboto Slab Light}}
\setbeamerfont{block title}{family=\fontspec{Roboto Slab}}
\setbeamerfont{block title alerted}{family=\fontspec{Roboto Slab}}
\setbeamerfont{block title example}{family=\fontspec{Roboto Slab}}

\setsansfont{Lucida Grande}
\usetheme{Rochester}

% Document
\begin{document}
\title[Template]{Beamer template}
\author{John SMITH}

\institute{Uppsala University, Sweden}
\date{\today}
\subject{Computer Science}

\frame[plain]{\titlepage}

\AtBeginSection[]{
    \begin{frame}
        \frametitle{Outline}
        \tableofcontents[currentsection, currentsubsection]
    \end{frame}
}

\AtBeginSubsection[]{
    \begin{frame}
        \frametitle{Outline}
        \tableofcontents[currentsection, currentsubsection]
    \end{frame}
}
\section{Section}
\subsection{Subsection}

\begin{frame}{Title of the frame}
    Some content about something, possibly with references \cite{REFERENCE}
    \begin{enumerate}
        \item First step of your process \footnotemark[1].
        \item Second step of your process.
    \end{enumerate}

    \begin{block}{What is a block}
        This is a block 
    \end{block}

    \begin{align*}
        \sum_{i=0}^{i=k}k+1 - \tau
    \end{align*}

    \footnotetext[1]{This is a footnote.}
\end{frame}

\begin{frame}{Table}
\begin{table}[ht]
\caption{A table with statistics}
\centering
\fontsize{10pt}{15pt}\selectfont
   \begin{tabularx}{\textwidth}{X|X|X}
      \textbf{Column1} & \textbf{Column2} & \textbf{Column3} \\
      \hline\hline
      Something 1 2014 & 10,3 \% (68) & 89,7 \% (591)\\
      Something 2 2014 & 19,6 \% (197) & 80,4 \% (809)\\
      Something 1 2013 & 9,9 \% (54) & 90,1 \% (489)\\
      Something 2 2013 & 18,9 \% (186) & 81,1 \% (782)\\
   \hline\hline
   \end{tabularx}
\end{table}

\end{frame}

\begin{frame}[fragile]{What if I want source code?}
    \begin{listing}[H]
        \begin{minted}[mathescape, % allows math symbols
                linenos, % number of lines
                bgcolor=darkgray, % Background color
            ]{haskell}
            -- Equations work as well: $\tau = \sum_{k \in s}$
            map :: (a -> b) -> [a] -> [b]
            map f (x:xs) = f x : map f xs
        \end{minted}
        \caption{Some Haskell source code}
        \label{lst:haskell}
    \end{listing}
\end{frame}

\section{References}
\bibliography{literature}
\bibliographystyle{IEEEtran}

\end{document}
